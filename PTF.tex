\documentclass[preprint,5p]{elsarticle}

%\usepackage{ecrc}
%% The ecrc package defines commands needed for running heads and logos.
%% For running heads, you can set the journal name, the volume, the starting page and the authors

%\volume{00}
%\firstpage{1}
%% Give the name of the journal
\journal{Nuclear Instruments and Methods}



%% The choice of journal logo is determined by the \jid and \jnltitlelogo commands.
%% A user-supplied logo with the name <\jid>logo.pdf will be inserted if present.
%% e.g. if \jid{yspmi} the system will look for a file yspmilogo.pdf
%% Otherwise the content of \jnltitlelogo will be set between horizontal lines as a default logo

%% Give the abbreviation of the Journal.
%\jid{procs}

%% Give a short journal name for the dummy logo (if needed)
%\jnltitlelogo{Procedia Computer Science}

%% Hereafter the template follows `elsarticle'.
%% For more details see the existing template files elsarticle-template-harv.tex and elsarticle-template-num.tex.

%% Elsevier CRC generally uses a numbered reference style
%% For this, the conventions of elsarticle-template-num.tex should be followed (included below)
%% If using BibTeX, use the style file elsarticle-num.bst

%% End of ecrc-specific commands
%%%%%%%%%%%%%%%%%%%%%%%%%%%%%%%%%%%%%%%%%%%%%%%%%%%%%%%%%%%%%%%%%%%%%%%%%%

%% The amssymb package provides various useful mathematical symbols
\usepackage{amssymb}
%% The amsthm package provides extended theorem environments
\usepackage{amsthm}

%% The lineno packages adds line numbers. Start line numbering with
%% \begin{linenumbers}, end it with \end{linenumbers}. Or switch it on
%% for the whole article with \linenumbers after \end{frontmatter}.
\usepackage{lineno}

%% natbib.sty is loaded by default. However, natbib options can be
%% provided with \biboptions{...} command. Following options are
%% valid:

%%   round  -  round parentheses are used (default)
%%   square -  square brackets are used   [option]
%%   curly  -  curly braces are used      {option}
%%   angle  -  angle brackets are used    <option>
%%   semicolon  -  multiple citations separated by semi-colon
%%   colon  - same as semicolon, an earlier confusion
%%   comma  -  separated by comma
%%   numbers-  selects numerical citations
%%   super  -  numerical citations as superscripts
%%   sort   -  sorts multiple citations according to order in ref. list
%%   sort&compress   -  like sort, but also compresses numerical citations
%%   compress - compresses without sorting
%%
\biboptions{comma,round,sort&compress,numbers}


% if you have landscape tables
\usepackage[figuresright]{rotating}

% put your own definitions here:
%   \newcommand{\cZ}{\cal{Z}}
%   \newtheorem{def}{Definition}[section]
%   ...

% add words to TeX's hyphenation exception list
\hyphenation{author another created financial paper re-commend-ed Post-Script}

% declarations for front matter

\begin{document}

\begin{frontmatter}

%% Title, authors and addresses

%% use the tnoteref command within \title for footnotes;
%% use the tnotetext command for the associated footnote;
%% use the fnref command within \author or \address for footnotes;
%% use the fntext command for the associated footnote;
%% use the corref command within \author for corresponding author footnotes;
%% use the cortext command for the associated footnote;
%% use the ead command for the email address,
%% and the form \ead[url] for the home page:
%%
\title{Photosensor Test Facility}


\author[ins:ubc,ins:triu]{S. Berkman}
\author[ins:ubc,ins:triu]{T. Feusels}
\ead{tfeusels@triumf.ca}
%\ead[url]{home page}
%\fntext[label4]{University of British Columbia}
\cortext[cor1]{}
\author[ins:triu]{W. Faszer}
\author[ins:triu]{D. Morris}
\author[ins:triu,ins:kav]{M. Hartz}
\author[ins:triu]{A. Konaka}
\author[ins:triu]{T. Lindner}
\author[]{B. Krupicz}
\author[]{A. Jaffray}
\author[]{H. Kugel}
\author[]{P. de Perlo}
\author[]{J. Linquiao Liu}
\author[]{P. Lu}
\author[]{A. Miller}
\author[]{C. Nantais}
\author[]{C. Reithmeier}
\author[]{C. Reithmeier}
\author[]{F. Reti\`ere}
\author[]{C. Reithmeier}
\author[]{C. Reithmeier}
\author[]{C. Reithmeier}
\author[]{C. Reithmeier}
\author[]{C. Reithmeier}
\author[]{C. Reithmeier}
\author[]{C. Reithmeier}



\address[ins:ubc]{University of British Columbia, Department of Physics and Astronomy, Vancouver, British Columbia, Canada}
\address[ins:triu]{TRIUMF, Vancouver, British Columbia, Canada}
\address[ins:kav]{Kavli Institute for the Physics and Mathematics of the Universe (WPI), The University
of Tokyo Institutes for Advanced Study, University of Tokyo, Kashiwa, Chiba, Japan}


%\dochead{Short communication}
%% Use \dochead if there is an article header, e.g. \dochead{Short communication}

%% use optional labels to link authors explicitly to addresses:
%% \author[label1,label2]{<author name>}
%% \address[label1]{<address>}
%% \address[label2]{<address>}

%UBC, TRIUMF and visitors: \\
%Sophie Berkman, Ben Krupicz, Patrick de Perio, Wayne Faszer, Tom Feusels, Mark Hartz, 
%Alexander Jaffray, Akira Konaka, Harish Kugel,
%Thomas Lindner, James Linqiao Liu, Philip Lu, Andy Miller, David Morris, Corina Nantais, 
%Yasuhiro Nishimura,
%Carl Reithmeier, Fabrice Retière, Sourav Sarkar, Mark Scott, Shaun Stephens-Whale, 
%Nils Smit-Anseeuw, Yusuke Suda, Hiro Tanaka, Shimpei Tobayama,
%Peter Vincent, Michael Walters, Michael Wilking, Stan Yen, Aaron Zimmer\\
%Y.Nishimura and Okajima
%Yusuke Koshio (Okayama Uni)
%Rika Sugimoto
%Takenaka akira
%Shigetaka MORIYAMA <moriyama@icrr.u-tokyo.ac.jp>
%Kevin Gin Sing Xie(Waterloo)
%Jashan Kaur RA (uni Winnipeg)

\author{}

\address{}

\begin{abstract}
%% Text of abstract
\end{abstract}

\begin{keyword}
%% keywords here, in the form: keyword \sep keyword

%% MSC codes here, in the form: \MSC code \sep code
%% or \MSC[2008] code \sep code (2000 is the default)

\end{keyword}

\end{frontmatter}

%%
%% Start line numbering here if you want
%%
\linenumbers

%% main text
\section{Introduction}

Photosensor Test Facility

What is the PTF? 



\section{Detailed study of Equipments}

%\begin{figure}
%  \centering
%  \includegraphics[width=4in]{gecko}
%  \caption[Close up of \species{Hemidactylus}]
%  {Close up of \species{Hemidactylus}, which is part the genus of the gecko family. It is the second most 
%    speciose genus in the family.}
%\end{figure}

%\newtheorem{name}{Printed output}



\subsection{Tank}

large grey plastic vessel

Should we line the inner surface? Stan has been investigating various
non-reflective surfaces

needs investigation into water compatibility and possible emanation





\subsection{Mechanical System}
frame

two gantries

use M11 control room






\subsection{Gantry System}

\subsection{Optical System/Head}

Shimpei

collimate/polarize the incoming light

watertight

how to manipulate?

new sources?


\subsection{Optisch Fibre}
ideal wavelength range is 350--650~nm

consider bending radius

throughput

tested three samples

selected FT200UMT

good transmission across the wavelengths tested~(378--777~nm)

seemed almost unaffected by curvature

fibre must reach several metres from laser through manipulator to the
optical head

how to mount, etc. to prevent tangles, excessive bending, etc.

\subsection{Magnetic Field Compensation}

assess magnetic shielding, magnetic compensation

GIRON magnetic shielding (http://www.lessemf.com/)

GIRON frame

Helmholtz coils, could include the dimensions, wire details,
measured resistance

could describe the testing we have done, eg. setting the voltage and
measuring the field with a handheld gaussmeter

Phidget on each gantry


coil positioning, maybe outside frame supports for coils, precision measurement of coil position

direction of current and maximum external fields

consider when synchrotron is on

control development

might have to do each scan more than once (debugging, etc.)

Magnetic field scans:
\begin{itemize}
\item{\bf Scan 0} no current, no GIRON, calculate the the fields that are needed from the coils
\item{\bf Scan 1} set currents
\item{\bf Scan 2} one layer of GIRON
\item{\bf Scan 3} two layers of GIRON
\end{itemize}


\subsection{Light Tightness/Dark room}

Stan

caulking

air vent light baffles

painting

curtains


\subsection{Water System}
Andy and Peter

Water circulation and filtration system

degassification is a major item


\subsection{DAQ/Controls}
\label{}

collision avoidance is in place and working

collision avoidance in the xz-plane will be necessary once a PMT is in place

further testing and refinement will take about 1 week

how to mount PMTs in the tank


\subsection{Photosensor}

measure different PMTs
\begin{itemize}
\item SK 20 inch
\item IceCube 13 inch, contact Darren Grant
\item hybrid photosensor
\end{itemize}

should contact Japanese, how many PMTs and when?

LBNE has made measurements

angular dependence

magnetic field dependence


measurements before the tank is filled with water

HV and signal must reach from high voltage power supply to the optical
head



\section{Operational Details}


from 20 February 2014, needs to be adjusted

\begin{table}[h]
\centering
\begin{tabular}{lc}
\toprule
Task & Time (d)\\
\midrule
software & 21\\
fix coils & 1\\
Scan 0 & 7 \\
Scan 1 & 4 \\
add one layer of GIRON & 1\\
Scan 2 & 4\\
add another layer of GIRON & 1\\
Scan 3 & 4\\
\midrule
{\bf Total} & {\bf 43}\\
\bottomrule
\end{tabular}
\label{tab:schedule}
\end{table}





%% The Appendices part is started with the command \appendix;
%% appendix sections are then done as normal sections
\appendix

\section{}
\label{}



%% References
%%
%% Following citation commands can be used in the body text:
%% Usage of \cite is as follows:
%%   \cite{key}         ==>>  [#]
%%   \cite[chap. 2]{key} ==>> [#, chap. 2]
%%

%% References with BibTeX database:

%\bibliographystyle{elsarticle-num}
%\bibliography{<your-bib-database>}

%% Authors are advised to use a BibTeX database file for their reference list.
%% The provided style file elsarticle-num.bst formats references in the required Procedia style

%% For references without a BibTeX database:

% \begin{thebibliography}{00}

%% \bibitem must have the following form:
%%   \bibitem{key}...
%%

% \bibitem{}

% \end{thebibliography}

\end{document}

%%
%% $Id: NIMpaper.tex,v 1.47 2011/06/23 02:49:38 alpinist Exp $
%%
%%
%\documentclass[preprint,12pt]{elsarticle-la}
% To get the author list and the institutional list correctly use this 

% Use this bug-fixed version for two column format 
%\documentclass[draft,5p,times,twocolumn,longtitle]{elsarticle-la}
\documentclass[final,5p,times,twocolumn,longtitle]{elsarticle-la}

\usepackage[dvips]{color}
\usepackage{NIMpaper}
\usepackage{float}
\usepackage{subfig}
\usepackage[abs]{overpic}
%%HEPunits.sty

%% Use the option review to obtain double line spacing
%\documentclass[preprint,review,number,sort&compress,12pt]{elsarticle}
\usepackage{lineno}
%\linenumbers
%% Use the options 1p,twocolumn; 3p; 3p,twocolumn; 5p; or 5p,twocolumn
%% for a journal layout:
%% \documentclass[final,1p,times]{elsarticle}
%% \documentclass[final,1p,times,twocolumn]{elsarticle}
%% \documentclass[final,3p,times]{elsarticle}
%% \documentclass[final,3p,times,twocolumn]{elsarticle}
%% \documentclass[final,5p,times]{elsarticle}
%% \documentclass[final,5p,times,twocolumn]{elsarticle}

%% if you use PostScript figures in your article
%% use the graphics package for simple commands
%% \usepackage{graphics}
%% or use the graphicx package for more complicated commands
\usepackage{graphicx}
%% or use the epsfig package if you prefer to use the old commands
%% \usepackage{epsfig}
\graphicspath{{/figures/}}
\DeclareGraphicsExtensions{{eps}{epsi}}

%% The amssymb package provides various useful mathematical symbols
\usepackage{amssymb}
%% The amsthm package provides extended theorem environments
%% \usepackage{amsthm}

%% The lineno packages adds line numbers. Start line numbering with
%% \begin{linenumbers}, end it with \end{linenumbers}. Or switch it on
%% for the whole article with \linenumbers after \end{frontmatter}.
%% \usepackage{lineno}

\usepackage{url}

%% natbib.sty is loaded by default. However, natbib options can be
%% provided with \biboptions{...} command. Following options are
%% valid:

%%   round  -  round parentheses are used (default)
%%   square -  square brackets are used   [option]
%%   curly  -  curly braces are used      {option}
%%   angle  -  angle brackets are used    <option>
%%   semicolon  -  multiple citations separated by semi-colon
%%   colon  - same as semicolon, an earlier confusion
%%   comma  -  separated by comma
%%   numbers-  selects numerical citations
%%   super  -  numerical citations as superscripts
%%   sort   -  sorts multiple citations according to order in ref. list
%%   sort&compress   -  like sort, but also compresses numerical citations
%%   compress - compresses without sorting
%%
%% \biboptions{comma,round}

% \biboptions{}

\journal{Nuclear Instruments and Methods in Physics}

\begin{document}

%\begin{frontmatter}

%% Title, authors and addresses

%% use the tnoteref command within \title for footnotes;
%% use the tnotetext command for the associated footnote;
%% use the fnref command within \author or \address for footnotes;
%% use the fntext command for the associated footnote;
%% use the corref command within \author for corresponding author footnotes;
%% use the cortext command for the associated footnote;
%% use the ead command for the email address,
%% and the form \ead[url] for the home page:
%%
%% \title{Title\tnoteref{label1}}
%% \tnotetext[label1]{}
%% \author{Name\corref{cor1}\fnref{label2}}
%% \ead{email address}
%% \ead[url]{home page}
%% \fntext[label2]{}
%% \cortext[cor1]{}
%% \address{Address\fnref{label3}}
%% \fntext[label3]{}

%% Add symbol footnote def
\long\def\symbolfootnote[#1]#2{\begingroup%
\def\thefootnote{\fnsymbol{footnote}}\footnote[#1]{#2}\endgroup}

\title{The PTF Paper}

%% use optional labels to link authors explicitly to addresses:
%% \author[label1,label2]{<author name>}
%% \address[label1]{<address>}
%% \address[label2]{<address>}

%\author{}
%\address{}
\input AuthorList_alphabetic.tex

%\begin{collab}
%(The T2K Collaboration)\\
%\end{collab}

\input institutionAddress.tex
%\input AuthorList_by_institution.tex


\begin{abstract}
The PTF does some stuff, which we should describe briefly here...
This paper provides a comprehensive review of the PTF facility and shows the results of the first PMT scans. 
\end{abstract}

\begin{keyword}
%% keywords here, in the form: keyword \sep keyword
Neutrinos \sep Neutrino Oscillation \sep Long Baseline \sep 
T2K\sep J-PARC \sep Super-Kamiokande
\PACS 14.60.Lm\sep 14.60.Pq\sep 29.20.dk\sep 29.40.Gx\sep 
      29.40.Ka\sep 29.40.Mc\sep 29.40.Vj\sep 29.40.Wk\sep 29.85.Ca 

%% MSC codes here, in the form: \MSC code \sep code
%% or \MSC[2008] code \sep code (2000 is the default)

\end{keyword}

%\end{frontmatter}

\maketitle

%%
%% Start line numbering here if you want
%%
%\linenumbers

%\twocolumn[]

%% main text
\section{Introduction}
\label{introduction} 
\input intro.tex
\section{Facility}
\label{facility}
\input facility.tex
\section{Photosensors}
\label{photosensors}
\input photosensors.tex
\section{Results}
\label{results}
\input results.tex
\section{Conclusion}
\label{conclusion}
\input conclusion.tex

%% The Appendices part is started with the command \appendix;
%% appendix sections are then done as normal sections
%% \appendix

%% \section{}
%% \label{}

%% References
%%
%% Following citation commands can be used in the body text:
%% Usage of \cite is as follows:
%%   \cite{key}         ==>>  [#]
%%   \cite[chap. 2]{key} ==>> [#, chap. 2]
%%

%% References with bibTeX database:

\bibliographystyle{./elsarticle/model1a-num-names}
\bibliography{NIMpaper}
%% Authors are advised to submit their bibtex database files. They are
%% requested to list a bibtex style file in the manuscript if they do
%% not want to use elsarticle-num.bst.

%% References without bibTeX database:

% \begin{thebibliography}{00}

%% \bibitem must have the following form:
%%   \bibitem{key}...
%%

% \bibitem{}

% \end{thebibliography}

%\end{multicols}
\end{document}

%%
%% End of file `elsarticle-template-num.tex'.
